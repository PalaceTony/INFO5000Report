\section{Results}
\subsection{Interpretation of Data}
\subsection{Comparative Study}
\subsubsection{Micro Level - Evaluation of sanitary napkin stations in Guangdong}
Explain our rating system and compare the 3 sanitary pad dispensers according to Excel.

\subsubsection{Macro Level - Compare China with other countries}
In 2021 during the 13th National People’s Congress conference, a member of the National People’s Congress named Wang, Zuoying suggested to the Treasury Department these three policies:
\begin{enumerate}
    \item the inclusion of the basic version of sanitary napkins into medical insurance;
    \item establish a designated fund to provide free period to certain group of women;
    \item promote the national reduction of value-added tax on feminine hygiene products.
\end{enumerate}
These policies were not sponsored, and the Treasury Department redirected the topic to the National Health Commission.\\

Internationally, Scotland became the first region to offer menstrual supplies free of charge in 2020. India, our less developed neighbor, abolished the tax on sanitary products in 2018 and is promoting a program of cheap sanitary napkins for one rupee (RMB 0.09) a piece. Korea doesn't have a nationwide program but started to distribute free sanitary napkins in 10 public areas in Central Seoul starting from 2018.\\

Table \ref{table:free_menstrual_supplies} is a summary of the 20 countries and regions that provide different levels of free menstrual supplies.


\begin{table*}[ht]
    \centering
    \caption{20 countries and regions that provide different levels of free menstrual supplies.}
    \begin{tabular}{|p{0.3\textwidth}|p{0.3\textwidth}|p{0.3\textwidth}|}
        \hline
        \textbf{Free to all in all public areas}                      & \textbf{No tax} & \textbf{Free in certain regions or public areas}                                                          \\ \hline
        Scotland (community centers, youth clubs, and pharmacies etc) & India           & New Zealand (school)                                                                                      \\ \hline
                                                                      & Kenya           & Australia (school in New South Wales and Victoria)                                                        \\ \hline
                                                                      & South Africa    & United States (school in Illinois, Washington, New York, New Hampshire, and Virginia)                     \\ \hline
                                                                      &                 & France (school in Île-de-France)                                                                          \\ \hline
                                                                      &                 & Kenya (school)                                                                                            \\ \hline
                                                                      &                 & South Africa (school)                                                                                     \\ \hline
                                                                      &                 & Botswana (school)                                                                                         \\ \hline
                                                                      &                 & Korea (public areas, e.g., library in Central Seoul)                                                      \\ \hline
                                                                      &                 & Uganda (school)                                                                                           \\ \hline
                                                                      &                 & Zambia (rural school)                                                                                     \\ \hline
                                                                      &                 & Canada (schools and federal agencies in British Columbia, Ontario, Nova Scotia, and Prince Edward Island) \\ \hline
    \end{tabular}
    \label{table:free_menstrual_supplies}
\end{table*}

\section{Challenges and Limitations}

\subsection{Survey Improvement }
The survey design is not yet complete and might causes bias. The way questions are framed influence how respondents interpret and answer them, leading results skew in a way that assumes certain preferences. For example, “no interest in buying other products” is not an option in the question to explore additional products.\\

Acquiescence bias can also be caused by a tendency to agree with statements regardless of their content, especially relating to personal habits (sanitary issue, price, conflict of interest, cultural issue, etc.) as well as the location selection.

\subsection{Vending Machine IoT System}
Discuss any challenges faced in designing the IoT system and purchasing software of the sanitary pad stations.\\

\subsection{Visulization Map Data Collection }
Address any limitations of your research methods or data.

