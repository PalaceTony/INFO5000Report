\section{Recommendations and Future Work}

\subsection{Leverage online supplier \& Lowering Cost}
To maintain a sustainable business, it is important to explore partnerships with popular online platforms for discounts or special offers for students and staff. Implement programs to provide free or subsidized sanitary products to students, especially those in financial need.

\subsection{Sustainable Design \& Environmental Awareness}
Promote the use of sustainable menstrual products like menstrual cups, cloth pads, or biodegradable pads. For example, we could add survey questions about awareness and usage of eco-friendly sanitary products to gauge environmental consciousness. Raise awareness about the environmental burden of sanitary waste disposal practices and implement proper disposal methods to minimize environmental impact. We have also been in touch with professors and students in the Material Science field to collaborate and potentially make sanitary napkins recyclable.

In addition, we aspire to attach solar panels to the vending machine to make it functional off-grid outside. This design is beneficial to under-developed rural areas in China. We could even attach it to tricycles and distribute it free of charge after securing donations. Big brands like Cola have already been using solar panel vending machines in Japan. Thus, it also has an overseas market.

\subsection{Health and Hygiene Education}
Organize workshops and seminars on menstrual health and hygiene. Include information on sustainable practices and proper disposal methods. This not only provides essential gender and health education for young girls and women in the community but also could raise awareness about women’s rights in society.

\subsection{Managing Conflicts of Interest}
\begin{enumerate}
    \item \textbf{Transparent Procurement Processes:} Ensure transparency in the procurement of sanitary products, avoiding any appearance of favoritism or conflicts of interest, and use open tenders and clear criteria for selecting suppliers.
    \item \textbf{Diverse Committee Involvement:} Form a committee with diverse representation (students, faculty, administrative staff) to oversee initiatives related to sanitary issues. This committee can also review potential conflicts of interest. Regarding the implementation of stations on campus and validating the rating system, we plan to consult experts such as staff members trained in electricity safety and handicap people association regarding their user experience.
    \item \textbf{Regular Audits and Reviews:} Conduct regular audits of procurement processes and initiatives. Ensure funds allocated for sanitary issues are used effectively and transparently.
\end{enumerate}

\subsection{Upholding University Image}
\begin{enumerate}
    \item \textbf{Positive Communication Strategy:} Publicize efforts and initiatives related to improving sanitary facilities and promoting gender inclusivity. Use the university's platforms to highlight commitments to sustainability and social responsibility.
    \item \textbf{Community Engagement:} Involve the university community in decision-making processes related to sanitary issues. Encourage feedback and suggestions from students and staff.
    \item \textbf{Collaboration with External Organizations:} Partner with NGOs, governmental organizations, and corporations that focus on menstrual health and gender equality. These collaborations can enhance the university's reputation as a socially responsible institution.
    \item \textbf{Regular Updates and Reporting:} Provide regular updates to the university community about ongoing initiatives and improvements. Transparency in actions and communications helps maintain a positive image.
\end{enumerate}

\subsection{Legislation Recommendations}
We plan to expand to other high schools, middle schools, and even primary schools. For the beneficiary students that do not have income, such as those younger than 18, it is likely that they cannot afford the education program or the sanitary napkin product.

We can find alternative `customers' such as those people affiliated with the school:
\begin{itemize}
    \item asking alumni, the investment office, or logistics office from the schools to sponsor an emergency sanitary napkin station;
    \item inviting the sanitary napkin producers to give the products to the schools;
    \item inviting local businesses to sponsor local schools to get the products and education program as a community goodwill project;
    \item asking donors to fund the sanitary napkins, e.g., through an online donation page at crowd-sourcing websites, Red Cross, or Tencent Charity.
\end{itemize}

We also would like to promote the inclusion of the basic version of sanitary napkins into medical insurance and promote the national reduction of value-added tax on feminine hygiene products. We plan to model the policy details after the example of the baby care room legislation in Guangzhou.

In the future, we will find our advocate from members of the National People’s Congress and the National Committee of the Chinese People’s Political Consultative Conference. We will collaborate with them and other locations that implemented the Emergency Feminine Product Accessible station to suggest to the Health Commission office in Guangzhou to share our experiences and demand legislative laws to mandate the delivery of sanitary napkins in public places.
