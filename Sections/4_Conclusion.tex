\section{Conclusion}
Our project “FreePeriod” uses teamwork, creative thinking, and resource connections to make sanitary pads available to those in need. We built an IoT vending machine prototype for our emergency geminie product accessible station network. We also built a website data visualization analyzing existing stations in Guangdong province. We further developed a rating system, conducted school user survey to implement the station on campus, and proposed multiple policy recommendations. By implementing these suggestions, any institution can effectively address period stigma and period poverty issues, manage potential conflicts of interest, and uphold a progressive and inclusive image. These steps not only improve the immediate environment for students and staff but also contribute to broader societal changes regarding menstrual health and gender equality.